\section{Technical Preliminaries}
\label{sect:prelims}

%%%%%%%%%%%%%%%%%%%%%%%%%%%%%
\subsection{Our \SSOT Variant}
\subsubsection{Our \SSOT variant}
\label{sec:ssotvar}
\subsubsection{Our \SSOT functionality}
\label{sec:ssotfunc}

In \figureref{fig:ssotfunc} we formally describe the variant \SSOT functionality we achieve. The sender has $n$ pairs $\{x_i, y_i\}$, the receiver has $m$ strings $x^*_j$. For each $j \in [1,m]$, the SSOT functionality allows the receiver to learn the $y_i$ if $x^*_j=x_i$, otherwise he learns a random string $\$$ that is randomly chosen by the sender.

\begin{figure}[htb]\centering
\framebox{
    \begin{minipage}{0.95\linewidth}
        The functionality is parameterized by a \SSOT $F$, and two parties: a {\bf sender} and {\bf receiver}.

        \medskip
				On input $n$ pairs $(\{x_1, y_1\},\{x_2, y_2\}, \ldots, \{x_n, y_n\})$ from the receiver, 
        On input $m$ strings $(x^*_1,x^*_2, \ldots, x^*_m)$ from the receiver, 
        \begin{itemize}
			\item Choose random strings: $r_1, r_2, \ldots, r_m$ and give these to the sender.
        	\item Give $(y^*_1,y^*_2, \ldots, y^*_m)$ to the receiver, where $y^*_j=y_i$ if $x^*_j=x_i$; $y^*_j=r_j$ if $x^*_j \not \in \{x_1, x_2, \ldots, x_m \}$				
				\end{itemize}

    \end{minipage}
}
\caption{The \SSOT ideal functionality.}
\label{fig:ssotfunc}
\end{figure} 


%%%%%%%%%%%
\subsection{PSI }
\subsubsection{PSI collusion}
\subsubsection{PSI functionality}
\label{sec:psifunc}
In \figureref{fig:psifunc} we formally describe the PSI functionality.% we achieve. The sender has $n$ pairs $\{x_i, y_i\}$, the receiver has $m$ strings $x^*_j$. For each $j \in [1,m]$, the SSOT functionality allows the receiver to learn the $y_i$ if $x^*_j=x_i$, otherwise he learns a random string $\$$ that is randomly chosen by the sender.

\begin{figure}[htb]\centering
\framebox{
    \begin{minipage}{0.95\linewidth}
    \todo{.....} \\
		\todo{...learn the intersection of $n$ sets, and nothing else..}  \\
		\todo{.....} 
    \end{minipage}
}
\caption{PSI ideal functionality.}
\label{fig:psifunc}
\end{figure} 

%%% Local Variables:
%%% mode: latex
%%% TeX-master: "main"
%%% End:
