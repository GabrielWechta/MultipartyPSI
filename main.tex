
\newif\iffullversion
\fullversionfalse
\fullversiontrue

\iffullversion
    \documentclass{article}
    \pagestyle{plain}
    \usepackage[letterpaper,margin=1in]{geometry}
\else
    \documentclass{sig-alternate}
\fi


\usepackage{xspace,amsmath,amsfonts,amssymb,hyperref,tikz,multirow}


\newcommand{\func}[1]{\ensuremath{\mathcal{F}_{\textsf{#1}}}\xspace}
\newcommand{\adv}{\ensuremath{\mathcal{A}}\xspace}

%%%%%%%%%%%%%%%%%%%%

\newcommand{\etal}{{\sl et~al.~}}
\newcommand{\eg}{{\sl e.g.}}
\newcommand{\ie}{{\sl i.e.}}
\newcommand{\apriori}{{\sl a~priori\/}\xspace}

\newcommand{\secpar}{\kappa}

\newcommand{\batchOPRF}{\textsf{BaRK-OPRF}\xspace}
\hyphenation{BaRK-OPRF}

\newcommand{\SSOT}{\todo{new} \textsf{SSOT}\xspace}
\hyphenation{SSOT}

\newcommand{\Enc}{\mathsf{Enc}}
\newcommand{\Dec}{\mathsf{Dec}}

%%%%%%%%%%%%%%%%%%%%%%

\newcommand{\rs}{\boldsymbol{r}}
\newcommand{\ts}{\boldsymbol{t}}
\newcommand{\us}{\boldsymbol{u}}
\newcommand{\qs}{\boldsymbol{q}}


%%%%%%%%%%%%%%%%%%%%%%

\newcommand{\algo}[1]{\ensuremath{\text{\sf #1}}\xspace}
\newcommand{\command}[1]{\ensuremath{\text{\sc #1}}\xspace}

%%%%%%%%%%%%%%%%%%%%

%% comment out for LNCS
\iffullversion
\usepackage{amsthm}
\newtheorem{theorem}{Theorem}
\newtheorem{definition}[theorem]{Definition}
\newtheorem{claim}[theorem]{Claim}
\newtheorem{lemma}[theorem]{Lemma}
\newtheorem{corol}[theorem]{Corollary}
\newtheorem{assumption}[theorem]{Assumption}
\newtheorem{obs}[theorem]{Observation}
\newtheorem{conj}[theorem]{Conjecture}
\fi

\newenvironment{proofof}[1]{\begin{proof}[Proof of #1.]}{\end{proof}}
\newenvironment{proofsketch}{\begin{proof}[Proof Sketch]}{\end{proof}}


%%%%%%%%%%%%%%%%%%%%%%%%%

\newcommand{\namedref}[2]{\hyperref[#2]{#1~\ref*{#2}}}
%% if you don't like it, use this instead:
%\newcommand{\namedref}[2]{#1~\ref{#2}}
\newcommand{\chapterref}[1]{\namedref{Chapter}{#1}}
\newcommand{\sectionref}[1]{\namedref{Section}{#1}}
\newcommand{\theoremref}[1]{\namedref{Theorem}{#1}}
\newcommand{\propositionref}[1]{\namedref{Proposition}{#1}}
\newcommand{\definitionref}[1]{\namedref{Definition}{#1}}
\newcommand{\corollaryref}[1]{\namedref{Corollary}{#1}}
\newcommand{\obsref}[1]{\namedref{Observation}{#1}}
\newcommand{\lemmaref}[1]{\namedref{Lemma}{#1}}
\newcommand{\claimref}[1]{\namedref{Claim}{#1}}
\newcommand{\figureref}[1]{\namedref{Figure}{#1}}
\newcommand{\subfigureref}[2]{\hyperref[#1]{Figure~\ref*{#1}#2}}
\newcommand{\equationref}[1]{\namedref{Equation}{#1}}
\newcommand{\appendixref}[1]{\namedref{Appendix}{#1}}


\definecolor{darkred}{rgb}{0.5, 0, 0} 
\definecolor{darkgreen}{rgb}{0, 0.5, 0} 
\definecolor{darkblue}{rgb}{0,0,0.5} 

\hypersetup{
    colorlinks=true,
    linkcolor=darkred,
    citecolor=darkgreen,
    urlcolor=darkblue   
}


%%%%%%%%%%%%%%%%%%%%%%%%%%%%%

\newcommand{\todo}[1]{%
    \mbox{}% prevent marginpar from being on previous paragraph
%    \marginpar{%
%        \colorbox{red!80!black}{\textcolor{white}{to-do}}%
%        \vspace*{-22pt}% hack!
%    }%
    \textcolor{red}{#1}%
}


    \renewcommand{\topfraction}{0.9}    % max fraction of floats at top
    \renewcommand{\bottomfraction}{0.8} % max fraction of floats at bottom


    %   Parameters for TEXT pages (not float pages):
    \setcounter{topnumber}{2}
    \setcounter{bottomnumber}{2}
    \setcounter{totalnumber}{4}     % 2 may work better
    \setcounter{dbltopnumber}{2}    % for 2-column pages
    \renewcommand{\dbltopfraction}{0.9} % fit big float above 2-col. text
    \renewcommand{\textfraction}{0.07}  % allow minimal text w. figs
    %   Parameters for FLOAT pages (not text pages):
    \renewcommand{\floatpagefraction}{0.7}      % require fuller float pages
        % N.B.: floatpagefraction MUST be less than topfraction !!
    \renewcommand{\dblfloatpagefraction}{0.7}   % require fuller float pages



\newif\ifsubmission
\submissionfalse

\ifsubmission
  \newcommand{\ch}[1]{#1}
  \newcommand{\del}[1]{}
\else
  \newcommand{\del}[1]{\textcolor{magenta}{[to be deleted: {\em {#1}}]}}
  \newcommand{\ch}[1]{\textcolor{blue}{#1}} 
\fi


\iffullversion
\let\oldparagraph\paragraph
\renewcommand\paragraph[1]{\oldparagraph{#1.}}
\else
\fi



\begin{document}


\iffullversion

\title{Practical Collusion-free Multi-party Private Set Intersection}

\iffalse

    \author{
        Vladimir Kolesnikov\thanks{Bell Labs, \textsf{kolesnikov@research.bell-labs.com}} \and
        Mike Rosulek\thanks{Oregon State University, \textsf{$\{$rosulekm,trieun$\}$@eecs.oregonstate.edu}} \and
        Ni Trieu\footnotemark[3]
    }
\fi
\else

\title{Practical Collusion-free Multi-party Private Set Intersection}
    \numberofauthors{3}
\iffalse

    \author{
    \alignauthor Vladimir Kolesnikov\\
           \affaddr{Bell Labs}\\
           \affaddr{Murray Hill, New Jersey}\\
           \email{kolesnikov@research.bell-labs.com}
    \alignauthor Mike Rosulek\\
           \affaddr{Oregon State University}\\
           \affaddr{Corvallis, Oregon}\\
           \email{rosulekm@eecs.\\oregonstate.edu}
    \alignauthor  Ni Trieu\\
           \affaddr{Oregon State University}\\
           \affaddr{Corvallis, Oregon}\\
           \email{trieun@eecs.\\oregonstate.edu}
    }

\fi

   \global\copyrightetc{Copyright is held by the owner/author(s). Publication rights licensed to ACM. 
    ACM 978-1-xxxxxxxxxxx  \\
    DOI: http://dx.doi.org/xxxxxxxxxxxxxxx
    }
\fi

%%%%%%%%%%%%%%%%%%%%%%%%%%%%%%%%%%%%%%%%%%%%%%%%%%%%%%%%%%%%%%%%%%%%%%%%%%%%%%
%
% macros.tex
%




%--
% Theorems are environments with numbering schemes. Based on Goldreich's defs.
%--

% \iffullversion
% \renewcommand\paragraph[1]{\bf{#1.}}
% \else
% \fi



\ifdefined \qed
\relax
\else
\newcommand{\qed}{\hspace*{\fill}\rule{7pt}{7pt}}
\fi

\ifdefined \theorem
\relax
\else
\newtheorem{theorem}{Theorem}
%\newtheorem{fact}{Fact}
\newtheorem{definition}{Definition}
\newtheorem{corollary}[theorem]{Corollary}
\newtheorem{example}{Example}
\newtheorem{lemma}[theorem]{Lemma}  % A counter for Lemmas etc
\newtheorem{proposition}[theorem]{Proposition}
\newtheorem{claim}[theorem]{Claim}

\newtheorem{conjecture}[theorem]{Conjecture}
\newenvironment{proof_sketch}{\quad\par\noindent{\bf Sketch of proof:~~}}{\qed\quad}
%\newenvironment{proof}{\noindent{\bf Proof:~~}}{\qed\quad}

\fi


\newtheorem{observation}[theorem]{Observation}
\newtheorem{fact}[theorem]{Fact}
\newtheorem{construction}{Construction}

\renewcommand\land{\ensuremath{\cdot}}
\renewcommand\odot{\ensuremath{\land}}

\newcommand\RF{\ensuremath{R}}   % ideal random function


\renewcommand\to{\ensuremath{\rightarrow}}
\newcommand\from{\ensuremath{\leftarrow}}
\newcommand\xor{\oplus}         %+ mod 2
\newcommand\Xor{{\bigoplus}}
\newcommand\poly{\mathop{\rm{poly}}\nolimits}
\newcommand\polylog{\mathop{\rm{polylog}}\nolimits}
%\newcommand\size{{\mathrm{size}}}
\newcommand\size[1]{\ensuremath{|#1|}}
\newcommand\sign{{\mathrm{sign}}}
\newcommand\bra{{\langle}}        %
\newcommand\ket{{\rangle}}        %
\newcommand\eps{{\epsilon}}        %
\newcommand\E{\mathop{\mathbb{E}}\displaylimits}   %expected value
\newcommand\Var{\mathop{\mathrm{Var}}\displaylimits}   %variance
\newcommand{\ceil}[1]{{\lceil{#1}\rceil}}
\newcommand{\floor}[1]{{\lfloor{#1}\rfloor}}
\newcommand{\set}[1]{\ensuremath{\{#1\}}}
\newcommand{\setsub}[2]{\ensuremath{\set{#1}_{#2}}}
\def\bool{\set{0,1}}
\newcommand{\angles}[1]{\langle #1 \rangle}
\newcommand{\pair}[2]{\langle #1,#2 \rangle}
\newcommand{\abs}[1]{\lvert#1\rvert}
\newcommand{\Abs}[1]{\left\lvert#1\right\rvert}
\newcommand{\nibf}[1]{\noindent \textbf{#1}}
\newcommand{\mednibf}[1]{\medskip\noindent{\bf #1}}


\newlength{\protowidth}
\newcommand{\pprotocol}[5]{
{\begin{figure*}[#4]
\begin{center}
\setlength{\protowidth}{.9\textwidth}
\fbox{
        \small
        \hbox{\quad
        \begin{minipage}{\protowidth}
    \begin{center}
    {\bf #1}
    \end{center}
        #5
        \end{minipage}
        \quad}
        }
        \caption{\label{#3} #2}
\end{center}
\vspace{-4ex}
\end{figure*}
} }


\newenvironment{CompactEnumerate}[1][enumi]{
  \begin{list}{\arabic{#1}.}{%
      \usecounter{#1} %
      \setlength{\topsep}{3pt}
      \setlength{\itemsep}{1pt}
      }}
  {\end{list}}
\newenvironment{CompactItemize}[1][enumi]{
  \begin{list}{--}{%
      \usecounter{#1}
      \setlength{\topsep}{3pt}
      \setlength{\itemsep}{1pt}
      }}
  {\end{list}}


\newlength{\saveparindent}
\setlength{\saveparindent}{\parindent}
\newlength{\saveparskip}
\setlength{\saveparskip}{\parskip}


\newcounter{ctr}
\newcounter{savectr}
\newcounter{ectr}

\newenvironment{tiret}{%
\begin{list}{\hspace{1pt}\rule[0.5ex]{6pt}{1pt}\hfill}{\labelwidth=15pt%
\labelsep=3pt \leftmargin=18pt \topsep=1pt%
\setlength{\listparindent}{\saveparindent}%
\setlength{\parsep}{\saveparskip}%
\setlength{\itemsep}{1pt}}}{\end{list}}

\newenvironment{eenum}{%
\begin{list}{{\rm (\arabic{ctr}.\arabic{ectr})}\hfill}{\usecounter{ectr}%
\labelwidth=28pt\labelsep=6pt \leftmargin=34pt \topsep=0pt%
\setlength{\listparindent}{\saveparindent}%
\setlength{\parsep}{\saveparskip}%
\setlength{\itemsep}{2pt} }}{\end{list}}

\newenvironment{newenum}{%
\begin{list}{{\rm \arabic{ctr}.}\hfill}{\usecounter{ctr}\labelwidth=17pt%
\labelsep=6pt \leftmargin=23pt \topsep=.5pt%
\setlength{\listparindent}{\saveparindent}%
\setlength{\parsep}{\saveparskip}%
\setlength{\itemsep}{5pt} }}{\end{list}}

\newenvironment{neweenum}{%
\begin{list}{{\rm \arabic{ctr}.\arabic{ectr}}\hfill}{\usecounter{ectr}%
\labelwidth=28pt\labelsep=4pt \leftmargin=34pt \topsep=0pt%
\setlength{\listparindent}{\saveparindent}%
\setlength{\parsep}{\saveparskip}%
\setlength{\itemsep}{2pt} }}{\end{list}}

\newenvironment{parenum}{%
\begin{list}{{\rm (\arabic{ctr})}\hfill}{\usecounter{ctr}\labelwidth=17pt%
\labelsep=6pt \leftmargin=23pt \topsep=.5pt%
\setlength{\listparindent}{\saveparindent}%
\setlength{\parsep}{\saveparskip}%
\setlength{\itemsep}{5pt} }}{\end{list}}

% general
\def\csp{\ensuremath{k}}
% OT Extension
\def\SS{\ensuremath{\mathcal{S}}}
\def\SSS{\ensuremath{\mathcal{S}^*}}
\def\RR{\ensuremath{\mathcal{R}}}
\def\ilen{\ensuremath{\ell}}
\def\numinst{\ensuremath{m}}
\def\sec{\ensuremath{s}}
\newcommand{\otpar}[2]{\ensuremath{\mathrm{OT}_{#2}^{#1}}}
\def\ot{\otpar{}{}}
\def\otkk{\otpar{\csp}{\csp}}
\def\otmk{\otpar{\numinst}{\csp}}
\def\otkm{\otpar{\csp}{\numinst}}
\def\otnm{\otpar{n}{\numinst}}
\def\otnk{\otpar{n}{\csp}}
\def\otml{\otpar{\numinst}{\ilen}}
\newcommand{\lotpar}[2]{\ensuremath{\mathrm{OTN}_{#2}^{#1}}}
\def\lot{\lotpar{}{}}
\def\lotml{\lotpar{\numinst}{\ilen}}
% iknp
\def\Q{\ensuremath{Q}}
\def\bq{\ensuremath{\boldsymbol{q}}}
\def\br{\ensuremath{\boldsymbol{r}}}
\def\qi{\ensuremath{\bq^i}}
\def\qj{\ensuremath{\bq_j}}
\def\Tz{\ensuremath{T_0}}
\def\To{\ensuremath{T_1}}
\def\bt{\ensuremath{\boldsymbol{t}}}
\def\tjz{\ensuremath{\bt_{0,j}}}
\def\tjo{\ensuremath{\bt_{1,j}}}
\def\tiz{\ensuremath{\bt^i_0}}
\def\tio{\ensuremath{\bt^i_1}}
\def\concat{\ensuremath{\|}}
\def\bs{\ensuremath{\boldsymbol{s}}}
\def\bss{\ensuremath{\bs^*}}
\def\sec{\ensuremath{\bs}}
\def\seci{\ensuremath{{s_i}}}
\def\ri{\ensuremath{r_i}}
\def\rj{\ensuremath{{r_j}}}
\def\ro{\ensuremath{r_1}}
\def\rm{\ensuremath{r_\numinst}}
% gen iknp
%\def\Del{\ensuremath{\Delta}}
\def\Del{\ensuremath{\mathcal{C}}}
%\newcommand{\delpar}[1]{\ensuremath{\mathbf{\delta}_{#1}}}
\newcommand{\delpar}[1]{\ensuremath{\boldsymbol{c}_{#1}}}
\def\delz{\delpar{0}}
\def\delo{\delpar{1}}
\def\xiz{\ensuremath{x_{i,0}}}
\def\xjz{\ensuremath{x_{j,0}}}
\def\xib{\ensuremath{x_{i,b}}}
\def\xij{\ensuremath{x_{i,j}}}
\def\yib{\ensuremath{y_{i,b}}}
\def\sib{\ensuremath{u_{i,b}}}
\def\zi{\ensuremath{z_i}}
\def\yiri{\ensuremath{y_{i,\ri}}}
\def\siz{\ensuremath{u_{i,0}}}
\def\sio{\ensuremath{u_{i,1}}}
\def\yiz{\ensuremath{y_{i,0}}}
\def\yio{\ensuremath{y_{i,1}}}
\def\siri{\ensuremath{u_{i,\ri}}}
\def\yjz{\ensuremath{y_{j,0}}}
\def\yjr{\ensuremath{y_{j,r}}}
\def\yji{\ensuremath{y_{j,i}}}
\def\yjn{\ensuremath{y_{j,{\L-1}}}}
\def\xjn{\ensuremath{x_{j,{\L-1}}}}
\def\xin{\ensuremath{x_{i,{\L-1}}}}
\def\xjr{\ensuremath{x_{j,r}}}
\def\xji{\ensuremath{x_{j,i}}}
\def\xio{\ensuremath{x_{i,1}}}
\def\xjo{\ensuremath{x_{j,1}}}
\def\yjo{\ensuremath{y_{j,1}}}
\def\yjrj{\ensuremath{y_{j,\rj}}}
\def\xjrj{\ensuremath{x_{j,\rj}}}
\def\xiri{\ensuremath{x_{i,\ri}}}
\def\xj{\ensuremath{x_j}}
\def\zj{\ensuremath{z_j}}
\newcommand{\bwand}[2]{\ensuremath{#1 \odot #2}}
%1-out-of-L
\def\L{\ensuremath{N}}
\def\logL{\ensuremath{n}}
\def\delL{\delpar{\L-1}}
\def\delz{\delpar{0}}

\def\delr{\delpar{r}}
\def\delrj{\delpar{\rj}}

% for our paper, we use code C instead of codewords, so redefine delr. 
\newcommand\Code{\ensuremath{C}}
\renewcommand\delr{\ensuremath{\Code(r)}}
\renewcommand\delrj{\ensuremath{\Code(r_j)}}
\def\delrr{\Code(r')}
\def\delwji{\Code(w_{j,i})}



\def\delij{\delpar{i,j}}
\def\deli{\delpar{i}}
\def\delj{\delpar{j}}

%%%%%%%%%%%%%
% Coding theory stuff

\def\bc{\ensuremath{\boldsymbol{c}}}
\def\codesize{\ensuremath{q}}
\def\wordlen{\ensuremath{p}}
\def\seppar{\ensuremath{d}}
\newcommand{\bcind}[1]{\ensuremath{\bc_{#1}}}
\def\bcaa{\bcind{\aa}}
\def\bcbb{\bcind{\bbb}}
\def\bcz{\bcind{0}}
\def\bcn{\bcind{\codesize}}
\def\bci{\bcind{i}}
\def\bcj{\bcind{j}}
\def\CC{\ensuremath{\mathcal{C}}}
\def\logcsp{\ensuremath{n}}

\def\walhad{\ensuremath{\mathrm{WH}(\alpha)}}
\newcommand{\whx}[1]{\ensuremath{\mathrm{WH}(#1)}}

\def\fc{\ensuremath{\chi}}
\def\qq{\ensuremath{q}}
\def\fdom{\ensuremath{\bool^\qq}}
\def\aa{\ensuremath{\alpha}}
\def\bbb{\ensuremath{\beta}}
\def\fcb{\ensuremath{\fc_\beta}}
\def\fca{\ensuremath{\fc_\aa}}
\def\fbca{\ensuremath{\mathrm{WH}_\aa}}
\def\fCC{\ensuremath{\widetilde \CC}}

%% Standardizing
%\def\L{\ensuremath{\csp}}
\def\N{\ensuremath{n}}
\def\whtext{Walsh-Hadamard}
\def\logcsp{\ensuremath{{\log\csp}}}
\def\wh{\ensuremath{\mathrm{WH}}}
\def\cwh{\ensuremath{\mathcal{C}_\wh}}
\def\cwhk{\ensuremath{\cwh^\csp}}

\def\L{\ensuremath{\N}}
\def\logL{\ensuremath{{\log \N}}}
\renewcommand{\lotpar}[2]{\ensuremath{{\L\choose 1}\mbox{-}\mathrm{OT}_{#2}^{#1}}}

\def\delk{\delpar{\csp-1}}

% PSI stuff
\newcommand{\msf}[1]{\ee{\mathsf{#1}}}











\maketitle

\begin{abstract}
We propose a PSI protocol for multi parties.... 

\end{abstract}



%%%%%%%%%%%%%%%
%%%ACM classifiers
% \iffullversion
% \else
% \begin{CCSXML}
% <ccs2012>
% <concept>
% <concept_id>10003752.10003777.10003788</concept_id>
% <concept_desc>Theory of computation~Cryptographic primitives</concept_desc>
% <concept_significance>500</concept_significance>
% </concept>
% <concept>
% <concept_id>10003752.10003777.10003789</concept_id>
% <concept_desc>Theory of computation~Cryptographic protocols</concept_desc>
% <concept_significance>500</concept_significance>
% </concept>
% </ccs2012>
% \end{CCSXML}

% \ccsdesc[500]{Theory of computation~Cryptographic primitives}
% \ccsdesc[500]{Theory of computation~Cryptographic protocols}

% \fi



% \begin{abstract}
% We introduce a new primitive that we call {\bf 1-out-of-Any Oblivious Transfer}. It allows a sender to query a random mapping $r : \{0,1\}^* \to \{0,1\}^\ell$ on any number of inputs while allowing a receiver to learn $r(x)$ for only a {\em single}, chosen string $x$. We first show how to efficiently construct this primitive by modifying existing (1-out-of-2) OT-extension protocols. The cost of the new primitive is only 3-6 times that of a random 1-out-of-2 (extended) OT, depending on whether semi-honest or malicious security is desired. 

% Next, we show an application of 1-out-of-any OT to private set intersection (PSI). The fastest state-of-the-art PSI protocol (Pinkas et al., Usenix 2015) is based on efficient OT extension. Our new primitive removes the protocol's dependence on the bit-length of the parties' items. We implemented the protocols and found ours to be \todo{2-10$\times$} faster.
% \end{abstract}


%%% Local Variables:
%%% TeX-master: "main"
%%% End:
% !TEX root = main.tex

\section{Introduction}
\label{sect:intro}

This work involves Oblivious Transfer(OT), String-Selection OT, OPRF, and PSI.  We start by reviewing the four primitives.

\paragraph{Oblivious Transfer}
\todo{need to paraphrase}

%Oblivious Transfer (OT) has been a central primitive in the area of secure computation.  Indeed, the original protocols of Yao~\cite{FOCS:Yao86} and GMW~\cite{GoldreichBook2,STOC:GolMicWig87} both use OT in a critical manner.  In fact, OT is both necessary and sufficient for secure computation~\cite{STOC:Kilian88}.  Until early 2000's, the area of generic secure computation was often seen mainly as a feasibility exercise, and improving  OT performance was not a priority research direction.  This changed when Yao's Garbled Circuit (GC) was first implemented~\cite{Fairplay} and a surprisingly fast OT protocol (which we will call IKNP) was devised by Ishai et al.~\cite{C:IKNP03}.
%
%The IKNP OT extension protocol~\cite{C:IKNP03} is truly a gem; it allows 1-out-of-2 OT execution at the cost of computing and sending only a few hash values (but a security parameter of public key primitives evaluations were needed to bootstrap the system).  IKNP was immediately noticed and since then universally used in implementations of the Yao and GMW protocols. It took a few years to realize that OT extension's use goes far beyond these fundamental applications.  Many aspects of secure computation were strengthened  and sped up by using OT extension.  For example, Nielsen et al.~\cite{C:NNOB12} propose  an approach to malicious  two-party secure computation, which relates outputs and inputs of OTs in a larger construction.  They critically rely on the low cost of batched OTs.
%Another example is the application of information-theoretic Gate Evaluation Secret Sharing (GESS)~\cite{AC:Kolesnikov05} to the computational setting~\cite{SCN:KolKum12}.  The idea of~\cite{SCN:KolKum12} is to stem the high cost in secret sizes of the GESS scheme by evaluating the circuit by shallow slices, and using OT extension to efficiently ``glue'' them together. Particularly relevant for our work, efficient OTs were recognized by Pinkas et al.~\cite{DBLP:conf/uss/Pinkas0Z14} as an effective building block for private set intersection, which we discuss in more detail later.
%
%The IKNP OT extension, despite its wide and heavy use, received very few updates.  In the semi-honest model it is still state-of-the-art.  Robustness was added by Nielsen~\cite{Nielsen07ot}, and in the malicious setting it was improved  only very recently~\cite{EC:ALSZ15,C:KelOrsSch15}.  Improvement for short secret sizes, motivated by the GMW use case, was proposed by Kolesnikov and Kumaresan~\cite{C:KolKum13}.   We use ideas from their protocol, and refer to it as the KK protocol.  Under the hood, KK~\cite{C:KolKum13} noticed that one core aspect of IKNP data representation can be abstractly seen as a repetition error-correcting code, and their improvement stems from using a better code.   As a result, instead of $1$-out of-$2$ OT, a $1$-out of-$n$  OT became possible at nearly the same cost,  for $n$ up to approximately $256$.

\todo{describe $1$-out of-$\infty$}


\paragraph{String Selection OT (Key-word search)}
\todo{we need this paragraph since receiver will get the sender's plaintext/random string based on his choice string}

\paragraph{Oblivious PRFs}

%An oblivious pseudorandom function (OPRF) is a protocol in which a sender
%learns (or chooses) a random PRF seed $s$ while the receiver learns $F(s,r)$, the result of the PRF on a single input $r$ chosen by the receiver. While the general definition of an OPRF allows the receiver to evaluate the PRF on several inputs, in this paper we consider only the case where the receiver can evaluate the PRF {\em on a single input.}
%
%
%The central primitive of this work, an efficient OPRF protocol, can be viewed as a variant of Oblivious Transfer (OT) of random values.  We  build it by modifying the core of OT extension protocols~\cite{C:IKNP03,C:KolKum13}, and its internals are much closer to OT than to prior works on OPRF.  Therefore, our presentation is OT-centric, with the results stated in OPRF terminology.
%       
%
%OT of random messages shares many properties with OPRF. In OT of random messages, the sender learns random $m_0, m_1$ while the receiver learns $m_r$ for a choice bit $r \in \{0,1\}$. One can think of the function $F( (m_0, m_1), r ) = m_r$ as a pseudorandom function with input domain $\{0,1\}$. Similarly, one can interpret 1-out-of-$n$ OT of random messages as an OPRF with input domain $\{1,\ldots,n\}$.
%
%\todo{describe BaRK-OPRF}


\paragraph{Two-Party Private Set Intersection}
%Private set intersection (PSI) refers to the setting where two parties each hold sets of items and wish to learn nothing more than the intersection of these sets.
%Today, PSI is a truly practical primitive, with extremely fast cryptographically secure implementations~\cite{DBLP:conf/uss/Pinkas0SZ15}. Incredibly, these implementations are only a relatively small factor slower than than the na\"{\i}ve and insecure method of exchanging hashed values.  Among the problems of secure computation, PSI is probably the one most strongly motivated by practice.  Indeed, already today companies such as Facebook  routinely share and mine shared information~\cite{effArticleFacebook,MotisCCStalk2015}.  In 2012, (at least some of) this  sharing was performed with insecure naive hashing.  Today, companies are able and willing to tolerate a reasonable performance penalty, with the goal of achieving stronger security~\cite{MotisCCStalk2015}.  We believe that the ubiquity and the scale of private data sharing, and PSI in particular, will continue to grow as big data becomes bigger and privacy becomes a more recognized issue.  We refer reader to~\cite{DBLP:conf/uss/Pinkas0SZ15,DBLP:conf/uss/Pinkas0Z14} for additional discussion and motivation of PSI.


\todo{describe PSI protocol[CCS:KKRT16]}

%\cite{CCS:KKRT16} significantly improves the PSI protocol of~\cite{DBLP:conf/uss/Pinkas0SZ15} by replacing one of its components with \batchOPRF. This change results in a factor 2.3--3.6$\times$ performance improvement for PSI of moderate-length strings (64 or 128 bits) and reasonably large sets \todo{......} 

\paragraph{NON-COLLUSION}
\todo{define non-collusion in MPC context}

\paragraph{Multi-party Private Set Intersection}

%%% Local Variables:
%%% TeX-master: "main"
%%% End:
% !TEX root = main.tex
\section{Related work}
\label{sect:relwork}

\subsection{Oblivious transfer}
\todo{need to paraphrase}

%As mentioned in Section~\ref{sect:intro}, given its critical importance in secure computation, the IKNP OT extension has a surprisingly short list of follow up improvements, extensions and generalizations. 
%
%Most relevant prior work for us is the KK protocol~\cite{C:KolKum13}, which views the  IKNP OT from a new angle and presents a framework generalizing IKNP.  More specifically, under the hood, players in the IKNP protocol encode Receiver's selection bit $b$ as a repetition string of $k$ copies of $b$.  KK generalized this and allowed the use of an error-correcting code (ECC) with large distance as the selection bit encoding.  For a code consisting of $n$ codewords, this allowed to do $1$-out of-$n$ OT with consuming a single row of the OT extension matrix.
%%In this work, we take the coding-theoretic perspective to the extreme.  We observe that we never need to decode codewords, and by using (pseudo-)random codes we are able to achieve what amounts to a 1-out-of-poly OT by consuming a single row of the OT matrix, which for the same security guarantee is only about $3.5\times$ longer than in the original IKNP protocol.

%Our work is strictly in the semi-honest security model. Other work on OT extension extends the IKNP protocol to the malicious model~\cite{EC:ALSZ15,C:KelOrsSch15} and the PVC (publicly verifiable covert) model~\cite{AC:KolMal15}.

\subsection{Oblivious PRF}
\paragraph{Oblivious PRF} 
%\sloppypar 

%\todo{need to paraphrase}
%
%Oblivious pseudorandom functions were introduced by Freedman, Ishai, Pinkas, \& Reingold~\cite{TCC:FIPR05}. In general, the most efficient prior protocols for OPRF require expensive public-key operations because they are based on algebraic PRFs.
%For example, an OPRF of \cite{TCC:FIPR05} is based on the Naor-Reingold PRF~\cite{NaoRei04} and therefore requires exponentiations. Furthermore, it requires a number of OTs proportional to the bit-length of the PRF input. 
%The protocol of \cite{EC:CamNevShe07} constructs an OPRF from unique blind signature schemes.
%The protocol of \cite{TCC:JarLiu09} obliviously evaluates a variant of the Dodis-Yampolskiy PRF~\cite{PKC:DodYam05} and hence requires exponentiations (as well as other algebraic encryption components to facilitate the OPRF protocol).

\paragraph{\batchOPRF functionality}


In \figureref{fig:oprf} we formally describe the variant OPRF functionality \cite{CCS:KKRT16}. It generates $m$ instances of the PRF with related keys, and allows the receiver to learn the (relaxed) output on one input per key.

\begin{figure}[htb]\centering
\framebox{
    \begin{minipage}{0.95\linewidth}
        The functionality is parameterized by a relaxed PRF $F$, a number $m$ of instances, and two parties: a {\bf sender} and {\bf receiver}.

        \medskip

        On input $(r_1, \ldots, r_m)$ from the receiver, 
        \begin{itemize}
			\item Choose random components for seeds to the PRF: $k^*, k_1, \ldots, k_m$ and give these to the sender.
            \item Give $\widetilde F( (k^*,k_1), r_1), \ldots, \widetilde F((k^*,k_m), r_m)$ to the receiver.
        \end{itemize}

    \end{minipage}
}
\caption{Batched, related-key OPRF (\batchOPRF) ideal functionality.}
\label{fig:oprf}
\end{figure}
\subsection{Private Keyword Search }
\label{sec:KS}

\subsection{Private set intersection}              
\subsubsection{Two-Party private set intersection}
%Oblivious PRFs have many applications, but in this paper we explore in depth the application to private set intersection (PSI).
%We consider only the semi-honest security model.  Our PSI protocol is
%most closely related to that of Pinkas et
%al.~\cite{DBLP:conf/uss/Pinkas0SZ15}, which is itself an optimized
%variant of a previous protocol of~\cite{DBLP:conf/uss/Pinkas0Z14}. We describe this protocol in great detail in \sectionref{sec:psi}.
%
%We refer the reader to \cite{DBLP:conf/uss/Pinkas0Z14} for an overview of the many different protocol paradigms for PSI. As we have mentioned, the OT-based protocols have proven to be the fastest in practice. We do, however, point out that the OT-based protocols do not have the lowest {\em communication} cost. In settings where computation is not a factor, but communication is at a premium, the best protocols are those in the Diffie-Hellman paradigm introduced in~\cite{DBLP:conf/sigecom/HubermanFH99}. In the semi-honest version of these protocols, each party sends only $2n$ group elements, where $n$ is the number of items in each set. However, these protocols require a number of exponentiations proportional to the number of items, making their performance slow in practice. Concretely, \cite{DBLP:conf/uss/Pinkas0SZ15} found Diffie-Hellman-based protocols to be over $200\times$ slower than the OT-based ones.
%
%While we closely follow the paradigm of \cite{DBLP:conf/uss/Pinkas0Z14}, we abstract parts of their protocol in the language of oblivious PRFs (OPRF). The connection between OPRF and PSI was already pointed out in \cite{TCC:FIPR05}. However, the most straightforward way of using OPRF to achieve PSI requires an OPRF protocol in which the receiver can evaluate the PRF on many inputs, whereas our OPRF allows only a single evaluation point for the receiver. OPRFs have been used for PSI elsewhere, generally in the malicious adversarial model \cite{TCC:JarLiu09,JC:HazLin10,TCC:Hazay15}.


\todo{describe PSI protocol[CCS:KKRT16]}


\subsubsection{Multi-Party private set intersection}
Privacy-preserving Set Operations \cite{C:KLD05}

Multi-Party Privacy-Preserving Set Intersection with
Quasi-Linear Complexity \cite{DBLP:journals/ieicet/CheonJS12}

Scaling Private Set Intersection to Billion-Element Sets \cite{KMES14}


\section{Technical Overview of Our Results}
\label{sect:overview}

\subsection{Programmable (Oblivious) PRFs}

\renewcommand{\P}{\mathcal{P}}

Let $\P$ be a set of points $\{ (x_i, y_i) \}$. A {\bf programmable PRF} $F$ satisfies two properties, informally:
\begin{itemize}
    \item Programmability: the output of $F$ will agree with the points $\P$, hence: $F(\P,k, x_i) = y_i$.
    \item Pseudorandomness: When the \todo{$x_i$ ?} values and key $k$ are chosen uniformly at random, then oracle access to $F(\P,k, \cdot)$ is indistinguishable from a random function. The actual definition has many caveats (see below), but the main spirit is that $F$ is pseudorandom.
\end{itemize}

An {\bf oblivious programmable PRF} protocol is a two-party protocol with the following behavior. The {\em sender} has a set of points $\P = \{ (x_i, y_i) \}$, and the receiver has a value $x^*$. The sender learns a PPRF key $k$, while the receiver learns $F(\P,k,x^*)$. Most importantly, the receiver does not learn anything about the programming set $\P$ (neither the $x_i$'s nor the $y_i$'s).

\paragraph{Caveats, disclaimers}
Our actual construction achieves properties that are slightly weaker than informally stated above. However, all of these relaxed notions are sufficent for our eventual application to PSI.
\begin{itemize}

    \item In the oblivious PPRF protocol, the receiver learns slightly more than $F(\P,k,x^*)$. This is in line with previous OPRF protocols \cite{xx,xx}. To formalize this, we say that the receiver learns a {\em relaxed PRF output} $\tilde F(\P,k,x^*)$, where: (1) the ``true'' output $F(\P,k,x^*)$ can be computed from the relaxed output, and (2) the pseudorandomness of $F$ holds with respect to an adversary who queries $\tilde F$. That is, outputs of $F$ look random to an adversary who learns some outputs of $\tilde F$. In the context of OPPRF, the relaxed output should also leak nothing about the programmed points $\P$.

    \item We require the pseudorandomness property to hold only against a bounded number of queries, and for adversaries who make a single query to the {\em relaxed} PRF $\tilde F$. Namely, for fixed $m$, an adversary who learns $\tilde F(\P,k,v^*)$ cannot distinguish $F(\P,k,v_1), \ldots, F(\P,k,v_m)$ from random (for any $\{ v_1, \ldots, v_m \} \not\ni v^*$). The parameters of the OPPRF depend on the bound $m$, which is a function of how OPPRF is used in an application.

    \item Our protocol generates many instances of OPPRF, each one with a different key. However, the keys are related. Specifically, we can write the key $k_i$ from the $i$th instance as $k_i = (k^*, k'_i)$, where $k^*$ is common to all instances. We require a PPRF that is secure in the presence of such related keys. Indeed, the PPRF that we use has this property.

\end{itemize}



\subsection{Application to multi-party PSI}

We consider the problem of private set intersection among many parties, in the semi-honest model.

\section{Technical Preliminaries}
\label{sect:prelims}

%%%%%%%%%%%%%%%%%%%%%%%%%%%%%
\subsection{Our \SSOT Variant}
\label{sec:ssotfunc}

In this section we formally describe the variant \SSOT definition. In \SSOT, the sender $\SS$  has $n$ distinct keywords with $n$ corresponding payloads $(x_i, p_i)$ and one dummy payload $\hat{p}$ that have to be indistinguishable from $p_i$. The receiver $\RR$ has a \textit{search word} $x^*$. If there is a keyword of the pair that is equal to the \textit{search word}, $\RR$  would receives the corresponding payload. Otherwise, $\RR$'s output is the dummy payload $\hat{p}$ created by Sender $\SS$ and $\RR$ does not know what happened.

The main difference of our \SSOT functionality with the \KS\cite{TCC:FIPR05} functionality is in that in \SSOT, $\SS$ also generate a dummy payload $\hat{p}$, and if the \textit{search word} is not in the set of the \textit{keyword}, $\RR$ receives the \textit{dummy} payload that is indistinguishable with the \textit{real} payload. In other words, unlike the \KS functionality\cite{TCC:FIPR05} in case that $\RR$ receives nothing if his \textit{search word} is not equal to any \textit{keyword} of $\SS$, in our \SSOT  $\RR$ always receives the secret string corresponding to his \textit{search word}.  The formal definition of our \SSOT is given in  
\figureref{fig:1ssotfunc}. For the privacy, our \SSOT requires both parties learns no additional information beyond their output that are defined by the definition in \figureref{fig:1ssotfunc}. 


\begin{figure}[htb]\centering
\framebox{
    \begin{minipage}{0.95\linewidth}
        The functionality is parameterized by a payload-length parameter $l$, and by two parties: a {\bf sender} and {\bf receiver}.

        \medskip
				   On input $n$ pairs $\{ (x_1, p_1), \ldots, (x_{n}, p_{n}) \}$ from the sender 
				(where $x_i \in \{0,1\}^*$ are all distinct and $p_i \in \{0,1\}^\ell$), \\
				 On input $x^* \in \{0,1\}^*$ from the receiver:
				\begin{itemize}
					\item Sender receives no output.
					\item If $x^* = x_i$ for some $i$, then give output $(n, p_i)$ to the receiver.
					\item Otherwise give output $(n,\hat{p})$ to the receiver, where $\hat{p}$ depends on $x^*$ and is indistinguishable with all $p_i$
				\end{itemize}

    \end{minipage}
}
\caption{The \SSOT ideal functionality}
\label{fig:1ssotfunc}
\end{figure} 

In \figureref{fig:1ssotfunc} definition, the Receiver $\RR$ searches for a single keyword. In fact, Receiver may repeat the key word procedure many times. To achieve an efficient solution in this case, we describe an extended version of \SSOT (called as \textit{non-adaptive}) that allows $\RR$ does multiple queries, but, at the same time. It means that our extended version limits these queries defined before running the procedure. An ideal functionality of the \textit{non-adaptive} \SSOT is given in \figureref{fig:nssotfunc}

\begin{figure}[htb]\centering
	\framebox{
		\begin{minipage}{0.95\linewidth}
			The functionality is parameterized by a payload-length parameter $l$, and by two parties: a {\bf sender} and {\bf receiver}.
			
			\medskip
			On input $n$ pairs $\{ (x_1, p_1), \ldots, (x_{n}, p_{n}) \}$ from the sender 
			(where $x_i \in \{0,1\}^*$ are all distinct, and $p_i \in \{0,1\}^\ell$), \\
			On input $m$ $\{x^*_1, \ldots, x^*_m\}$ from the receiver, (where $x_i \in \{0,1\}^*$):
			\begin{itemize}
				\item Sender receives no output.
				\item For each $ 1 \leq j \leq m$:
				\begin{itemize}
					
				\item if $x^*_j = x_i$ for some $i$, then give output $(n, p_i)$ to the receiver.
				\item Otherwise give output $(n,\hat{p_j})$ to the receiver, where $\hat{p_j}$ depends on $x_j^*$ and is indistinguishable with all $p_i$
			\end{itemize}
			\end{itemize}
			
		\end{minipage}
	}
	\caption{The non-adaptive $m$-times \SSOT ideal functionality}
	\label{fig:nssotfunc}
\end{figure} 



%%%%%%%%%%%
\subsection{PSI }
\subsubsection{PSI collusion}
\subsubsection{PSI functionality}
\label{sec:psifunc}
In \figureref{fig:psifunc} we formally describe the PSI functionality.% we achieve. The sender has $n$ pairs $\{x_i, y_i\}$, the receiver has $m$ strings $x^*_j$. For each $j \in [1,m]$, the SSOT functionality allows the receiver to learn the $y_i$ if $x^*_j=x_i$, otherwise he learns a random string $\$$ that is randomly chosen by the sender.

\begin{figure}[htb]\centering
\framebox{\begin{minipage}{0.95\linewidth}
		
		The functionality is parameterized by the number of parties $n$,  the size of the parties' sets $m$, and the bit-length of the parties' items $\sigma$. 
		\begin{itemize}
			\item On input $X_i=\{x^1_i, \ldots, x^m_i\} \subseteq\{0,1\}^\sigma$ from each party $P_i$.
			
			\item Give output $\bigcap\limits_{i=0}^n X_i$ to $P_i$.
		\end{itemize}
	\end{minipage}
}
\caption{PSI ideal functionality.}
\label{fig:psifunc}
\end{figure} 

%%% Local Variables:
%%% mode: latex
%%% TeX-master: "main"
%%% End:

%!TEX root = main.tex

\section{\SSOT construction}

We present our constructions as follows.  We start with the default \SSOT  (1-out-of-$n$ KS, defined in \figureref{fig:1ssotfunc}).  Next, in Section~\ref{sect:constrnSSOT} we give a protocol for $m$-times  \SSOT  in non-adaptive settings ($m$-out-of-$n$ KS, defined in \figureref{fig:nssotfunc}). We also prove the protocols secure against semi-honest adversaries.
 

\subsection{Notation}
\subsection{\SSOT construction}
\label{sect:constr1SSOT}

We construct a \SSOT protocol based on batched related-key oblivious pseudo-random function (\batchOPRF) \cite{CCS:KKRT16}. Recall the  \batchOPRF functionality that the receiver provides an input $x^*$; the functionality $F$ choose a random key $k$, gives $k$ to the sender and $F_k(x^*)$ length $\sigma=\sigma(\kappa)$ to the receiver. In other words, the sender can compute $F_k(x_i)$ for any $x_i$ while the receiver learns only $F_k(x^*)$ for a single value $x^*$. 

The basic idea of the \SSOT construction is that the sender $\SS$ uses the output of $F_k(x_i)$ as a key $sk_i$ to encrypt the corresponding payload $p_i$, and send to $\RR$ each encryption string $\Enc(sk_i,p_i)$ together with RO hash of the key used in this encryption.  The $\RR$ computes the RO hash of his key $F_k(x^*)$ and may obtain a single key while he will have no information to guess other keys. He may decrypt a cipher-text corresponding to his key $sk^*=F_k(x^*)$ and get the real payload.   This technique is similar to OT extension protocols  \cite{C:IKNP03,C:KolKum13}. However, $\RR$ may not be able to decrypt the corresponding cipher-text in case of that his search word is not in the sender's set of keyword since he does not have the key corresponding to the RO hash received from $\SS$, thus he may not decrypt any the cipher-text. It means that the receiver outputs nothing which is not true in the \SSOT functionality. Therefore, instead of sending RO hash of the encrypted key $sk_i$, the sender does the following steps:
\begin{itemize}
	\item look at $n$ encrypted keys $sk_i=F_k(x_i)$ length $\sigma=\sigma(\kappa,n)=\kappa+m$, where $m=\floor{\log(n)}+1$, choose $m$ bit positions $B=\{b_1,  \ldots, b_m\}$, under which the keys are distinct.%. Note that the length of $F$ is larger than $m$, thus it is able to find the set $B$. For example, length $|F|=128$ while the number of pairs $n=2^{24} \Rightarrow m=\floor{\log(n)}+1=25$.
	\item run a mapping function $M: \bool^\sigma \rightarrow \bool^m$ that represents each $sk_i$ in $m$ positions defined by $B$, \textit{i.e.}, $\tilde{sk}_{i}=M(sk_i)$
	\item generate $(2^m-n)$ strings $\tilde{sk}_{n+i}$ length $m$ to get $2^m$ distinct values $\tilde{sk}_{i}$ length $m$
	\item for $1\leq i \leq n$, $\SS$ sends RO hash of $\tilde{sk}_{i}$ along with each encryption string $y_i\Enc(sk_i,p_i)$. For $n < i \leq 2^m$, $\SS$ sends RO hash of $\tilde{sk}_{i}$ along with $y_i$, where $y_i$ is random string.
	\item send the function $M$, and set $B$	to $\RR$
\end{itemize}


By adding on these above steps, $\RR$ knows function $M$ and the set $B$, he can compute $\tilde{sk^*}=M(sk^*)$, and always gets the string corresponding to his key. Note that if his search word is in the sender's set of keyword $\{sk_1, \ldots , sk_n\}$, then he can decrypt a \textit{right} cipher-text using $sk^*$, and get the \textit{real} payload. Otherwise, he decrypt a random string created by $\SS$.


Moreover, we require an encryption scheme such that for all $m$, the distribution $Enc(sk, m)$ (induced by random choice of $sk$), is pseudorandom. In other words,the encryption scheme is one-time secure with a random choice of the encrypted key. Our \SSOT protocol is presented in \figureref{fig:Cons1ssot}. 

\begin{figure}[h]\centering
	\framebox{
		\begin{minipage}{0.95\linewidth}
			\noindent{
				\\
				{\sc Input of \SS:} $n$ pairs  
				$\{ (x_0, p_0), \ldots, (x_{n-1}, p_{n-1}) \}$, where $x_i \in \{0,1\}^*$, $x_i \ne x_j$,  and $p_i \in \{0,1\}^\ell$.
				
				\medskip
				
				{\sc Input of \RR:} selection strings $x^* \in \bool^*$.
				
				\medskip
				
				{\sc Parameters:}
				\begin{itemize}\addtolength{\itemsep}{-6pt}
					\item  A \batchOPRF function $F_k$ with output length $\sigma$, where $k$ is random key.
					\item A mapping function $M:\bool^\sigma \to \bool^m$, where $m=\floor{\log(n)}+1$ as defined above.
					\item A random hash  $H:\bool^m \to \bool^\kappa$, 
					\item A suitable encryption scheme $\Enc$ as defined above.
				\end{itemize}
			}
			
			{\sc Protocol:}
			
			\begin{enumerate}
				\addtocounter{enumi}{0}
				\item the parties performs a \batchOPRF with sender \SS\ and receiver \RR\
				\begin{enumerate}
					\item \SS\ receives a random key $k$, 					
					\item \RR\ receives $sk^*=F_k(x^*)$
					\item $\forall$ $0\leq i < \L$, \SS\ computes $sk_i = F_k(x_i)$.
				\end{enumerate}
				
				\item \SS\ chooses $m$ bit positions at random $B=\{b_1,  \ldots, b_m\}$, under which the $sk_i , i \in [0,\L],$ are distinct, sends it to \RR 
				
				\item 
				\begin{enumerate}
					\item $\forall$ $0\leq i < \L$, \SS\ computes $\tilde{sk}_{i} = M(sk_i)$
					\item $\forall$ $\L \leq i < 2^m$, \SS\ chooses $\tilde{sk}_{i} \from\bool^m$ at random, such that all $2^m$ values $\tilde{sk}_{i}, i \in [0, 2^m-1]$, are distinct.				
				\end{enumerate}
				\item  
				\begin{enumerate}
					\item $\forall$ $0\leq i < \L$, \SS\ computes $y_i = \Enc(sk_i, p_i)$
					\item $\forall$ $\L \leq i < 2^m$, \SS\ choose $y_i \from\bool^\kappa$ at random
				\end{enumerate}
				\item $\forall$ $0\leq i < 2^m$, \SS\ sends $\{H(\tilde{sk}_{i}), y_i\}$ to \RR\
				
				\item \RR\ receives $Y=\{H(\tilde{sk}_{i}), y_i\}$, $i\in[2^\numinst]$ and does the following: 
				\begin{enumerate}
					\item computes $\tilde{sk^*} = M(sk^*)$
					\item find $i$ such that $H(\tilde{sk^*})= H(\tilde{sk_i})$
					\item output $p=\Dec(sk^*, y_i)$
				\end{enumerate}				
				
			\end{enumerate}				
		\end{minipage}
	}
	\caption{1-out-of-$n$ \SSOT protocol}
	\label{fig:Cons1ssot}
\end{figure}


\begin{theorem} (Correctness)
	\label{thm:sotcorr}
In semi-honest setting, on the sender \SS\ inputs $n$ pairs $\{ (x_0, p_0), \ldots, (x_{n-1}, p_{n-1}) \}$, where $x_i \in \{0,1\}^*$, $x_i \ne x_j$,  and $p_i \in \{0,1\}^\ell$, the receiver \RR\ inputs $x^*$. After running protocol in \figureref{fig:Cons1ssot}, \RR\ outputs $p_i$ if $x^*=x_i$, or random value $\hat{p}$ if no such $i$ exists, where $\hat{p}$ depends on $x^*$ and is indistinguishable with all values $p_i$
\end{theorem}
\begin{proof}
	Let's consider two following cases:
	\begin{enumerate}
		\item exists $i \in [0, \L-1]$ such that $x^*=x_i$: it is easy to see from \batchOPRF that $F_k(x^*)=F_k(x_i) \Rightarrow sk^* = sk_i$. Thus, a RO hash  of $sk^*$ is equal to $H(\tilde{sk_i})$. \RR\  obtains the corresponding ciphertext, decrypts it using his key $sk^*=F_k(x^*)=sk_i$, and gets the corresponding payload $p_i$.
		
		\item no exist $i \in [0, \L-1]$ such that $x^*=x_i$: from the properties of \batchOPRF protocol, for all $0 \leq i < \L$,  $F_k(x^*) \ne F_k(x_i) \Rightarrow sk^* \ne sk_i \Rightarrow \tilde{sk^*} \ne \tilde{sk_i}$. However, \SS\ chooses $2^m-n$ random $sk_i$ in step 3(b) to fill out all cases of value $\tilde{sk_i}$. Note that length of $sk_i$ is $m$. There exist $i \in [\L, 2^m-1]$ such that $\tilde{sk^*}  = \tilde{sk_i}$. Therefore, \RR\  obtains the corresponding random ciphertext generated by step 4(b) , decrypts it using his key $sk^*=F_k(x^*)=sk_i$, and gets the random payload $\hat{p}$ which depends on value $x^*$. Since the distribution of $\Enc$ scheme is pseudorandom, $\hat{p}$ is indistinguishable with all values $p_i$.
	\end{enumerate}
Hence, the protocol is correct.
\end{proof}


In order to show that both parties do not learn any additional information more than in the ideal functionality defined by \ref{fig:1ssotfunc}, our protocol must satisfy two following properties:
\begin{itemize}
	\item the view of a semi-honest sender \SS\ are indistinguishable in the case that \RR\  inputs $x$ and the case that his input is $x'$.  
	\item  the view of semi-honest receivers \RR\ in the real protocol and his view in the ideal protocol on any inputs $\{ (x_0, p_0), \ldots, (x_{n-1}, p_{n-1}) \}$ of the sender \SS\ are indistinguishable. Especially, the view of \RR\ in case that his search word is in the keyword set of \SS\ and it is \textit{not} in this set, are indistinguishable.
\end{itemize}


\begin{theorem}(Privacy)
\label{thm:sotpriv}
The \SSOT protocol in \figureref{fig:Cons1ssot} securely realizes the  functionality of \figureref{fig:1ssotfunc} in the presence of semi-honest adversaries, provided that:
\begin{itemize}\addtolength{\itemsep}{-6pt}
	\item  A \batchOPRF function $F_k$ with random key $k$ and output length $\kappa$.
	\item A mapping function $M:\bool^\kappa \to \bool^m$, where $m=\floor{\log(n)}+1$ as defined above.
	\item A $\kappa$-RO hash  $H:\bool^m \to \bool^\kappa$, 
	\item A one-time secure encryption scheme $\Enc$.
\end{itemize}
Here,  $\kappa$ is the computational security parameter.
\end{theorem}

\begin{proof}

\medskip
\noindent{\bf Simulating \SS.} 
According the privacy of the \batchOPRF protocols~\cite{CCS:KKRT16}, it is trivial to argue that the view of a semi-honest \SS\ can be perfectly simulated. Indeed, the view of \SS\ in the protocol is a random key $k$ received from the \batchOPRF and uniformly random OPRF values $F_k(x_i)$. Therefore, the view of a semi-honest sender \SS\, in the case that \RR\  uses $x$ and the case that he uses  $x'$, are indistinguishable. 

\medskip
\noindent{\bf Simulating \RR.} On the view of a semi-honest receiver \RR\ in the protocol, the only external messages it receives are the \batchOPRF output $sk^*=F_k(x^*)$ at step 1(a), the $m$ bit positions $B=\{b_1,  \ldots, b_m\}$ that \RR\ knows all first $n$ encrypted keys $sk_i$ are distinct in step 2, and the set Y in step 6. Therefore, the simulated view is as follows:
 \begin{itemize}
 	\item Call the ideal \SSOT functionality with input $x^*$. Let $p$ denote as the output received from ideal-world trusted party TP.
 	
 	\item Call the ideal \batchOPRF functionality with input $x^*$. Let $sk^*=F_k(x^*)$ denote as the output received from ideal-world trusted party TP.
 	
 	
 	\item Chooses $m$ bit positions at random $B=\{b_1,  \ldots, b_m\}$ and send it to \RR\ on behalf of $\SS$ in step 2.
 	
 	\item In step 6, simulate the $Y$ messages from $\SS$ as follows:
 	compute  $\tilde{sk^*} = M( sk^*)$, and $y_i = \Enc( sk^*, p)$, and set a pair $\{H(\tilde{sk^*}), y_i \}$ to be a random permutation of $Y$ and $2^m-1$ other random pairs.
 	
 \end{itemize}

To prove that this simulated view is indistinguishable from the real one, we first consider the value that \SS\ uses as a key to $\Enc$. This encrypted key is the output of \batchOPRF, thus it is a pseudorandom string. However, the protocol also gives knowledge of the RO hash of $m$ bit string of the key as $H(M(sk_i))$. Since the output length of \batchOPRF is $\sigma=\kappa+m$ in our protocol, and we assume that $H(M(sk_i))$ leaks $m$ bit of the key $sk$, the probability for \RR\ learning the $\SS$'s key is $1/2^\kappa$. Therefore, choosing $m$ bit positions in step 2 can be simulated independently on the input of $\SS$. In other words, step 1(a), step 2, and step 6(a) are perfectly simulated.

Now, consider the encrypted scheme $\Enc$,  we require the distribution of $\Enc$ is pseudorandom, thus the real view is indistinguishable from the one in which value of the form  $\Enc(sk_i, p_i)$ in step 4(a) are replaced with random string in step 4(b). 

From the above observation, if the search word $x^*$ is not in the keyword set of \SS\, then the corresponding ciphertext is chosen uniformly. The security of $\Enc$ implies that these ciphertexts are indistinguishable from random strings. In addition, the encrypted key is indistinguishable in the case that search word is in the keyword set or not.  Hence, we see that the simulated view is indistinguishable from the real view, and the protocol is secure.
\end{proof}

\subsection{Non-adaptive \SSOT construction}
\label{sect:constrnSSOT}
We now present the non-adaptive \SSOT construction which allows the receiver does multiple keyword search queries, but the queries must defined before running the protocol. Since our \SSOT protocol is based on \textit{batch} related-key OPRF (\batchOPRF), this protocol shows an efficient solution to reduce the costs in this non-adaptive setting, and have application to our multi-party private set intersection described in \sectionref{sec:psi}. The basic idea of the construction is to use the \textit{hashing to bin} technique that is very popular in two-party PSI protocols~\cite{DBLP:conf/uss/Pinkas0SZ15,DBLP:conf/uss/Pinkas0Z14}. Our  non-adaptive \SSOT protocol is presented in \figureref{fig:Consnssot}. 


\begin{figure}[h]\centering
	\framebox{
		\begin{minipage}{0.95\linewidth}
			\noindent{
				\\
				{\sc Input of \SS:} $n$ pairs  
				$\{ (x_0, p_0), \ldots, (x_{n-1}, p_{n-1}) \}$, where $x_i \in \{0,1\}^*$, $x_i \ne x_j$,  and $p_i \in \{0,1\}^\ell$.
				
				\medskip
				
				
				{\sc Input of \RR:} $m$ selection strings $\{x^*_1, \ldots, x^*_m\}$, where $x_i \in \{0,1\}^*$.
				
				
				\medskip
				
				{\sc Parameters:}
				\begin{itemize}\addtolength{\itemsep}{-6pt}
					\item   1-out-of-n \SSOT function $F$ 
					\item three random hash $\{H_1, H_2, H_3\}: \{ 0,1\}^* \to [1.2m]$ 
					\item  max. stash size $s$ as defined in Table~\ref{tbl:params}
					\item max. number of items in $\SS$'s bin $\gamma$, where $\gamma$ as defined in Table~\ref{tbl:params}
				\end{itemize}
			}
			
			{\sc Protocol:}
			
			\begin{enumerate}
				\addtocounter{enumi}{0}
				\item \RR\ hashes his items $X^*=\{x^*_0, \ldots, x^*_{m-1}\}$ into $1.2m$ bins using Cuckoo hashing defined by 3 hash functions $\{H_1, H_2, H_3\}$. After the certain number of times, if the insertion attempts of the item $x^*_i$ to $[1.2n]$ bins  fails,  \RR\ put this item is in the stash bins in an arbitrary order; otherwise $x^*_i$ is in the bin \#$b(x^*_i) \in [1.2n]$.  
				\\
				For $i \in [1.2m+s]$, if bin \#$i$ is empty, then pad this bin using a dummy value $r_i$; otherwise if $x^*_j$ is in bin \#$i$ then set $r_i = x^*_j$. 
				
				\item \SS\ hashes his items $X=\{x_0, \ldots, x_{n-1}\}$ into $1.2m$ bins using 3 hash functions $\{H_1, H_2, H_3\}$. For $i \in [1.2m]$, index all $x_j$ is in bin \#$i$ as $r_{i,j}$, and pad this bin \#$i$ to $\gamma$ items using distinct dummy values $r_{i,k}$; for $i \in [s]$, pad this bin \#$i$ by the set $X$.
				
				\item For $i \in [1.2m]$  the parties performs a 1-out-of-n \SSOT protocol with sender \SS\ and receiver \RR\
			\begin{enumerate}
				\item \RR\ inputs a search word $r_i$, 					
				\item \SS\ inputs $\gamma$ pairs $(r_{i,j}, \hat{p_j})$, where $\hat{p_j}$ is equal to $p_j$ if $r_{i,j}$ is assigned as $x_j$; otherwise $\hat{p_j}$ is random.					
				\item \RR\ outputs $p_j$ if $r_i=x_j$; otherwise, \RR\ receives a dummy value. In other words, if \RR\ receives a payload corresponding to his search word in this bin.
			\end{enumerate}
							
				\item For $i \in [s]$  the parties performs a 1-out-of-n \SSOT protocol with sender \SS\ and receiver \RR\
				\begin{enumerate}
					\item \RR\ inputs a search word $r_i$, 					
					\item \SS\ inputs $n$ pairs $(x_i, p_i)$
					\item \RR\ outputs $p_j$ if $r_i=x_j$; otherwise, \RR\ receives a dummy value. 		
				\end{enumerate}
			
			\end{enumerate}				
		\end{minipage}
	}
	\caption{$m$-out-of-$n$ non-adaptive \SSOT protocol}
	\label{fig:Consnssot}
\end{figure}

\begin{table}\centering
	\framebox{
		\begin{minipage}{0.9\linewidth}\centering
			\begin{tabular}{c | c c c }
				$n$ &  $\beta$ & $s$ \\
				\hline
				$2^8$    & 24& 12  \\
				$2^{12}$ & 25& 6  \\
				$2^{16}$ & 26& 4 \\
				$2^{20}$ & 27& 3  \\
				$2^{24}$ &28 & 2  \\
			\end{tabular}
		\end{minipage}
	}
	
	\caption{Parameters used in \figureref{fig:Consnssot}. $n$ is the size of the parties' input sets; $\beta$ is the maximum bin size for simple hashing; $s$ is the maximum hash size for Cuckoo hashing}
	\label{tbl:params}
\end{table}

The correctness of the non-adaptive \SSOT protocol can be trivially achieved from \theoremref{thm:sotcorr}. For the security, the main difference of the construction with the above 1-out-of-n \SSOT construction is in that in the non-adaptive setting, \SS\ uses Simple hashing and \RR\ uses Cuckoo hashing to assign their item to the bin before running 1-out-of-n \SSOT protocol. The security of the hashing to bin technique had been proved in ~\cite{eprint:PSZ16} when using the parameters defined in Table~\ref{tbl:params}. Therefore, we can skip the proof of correctness and privacy of the protocol in \figureref{fig:Consnssot}



%\begin{theorem}(Privacy)
%	\label{thm:nsotpriv}
%	The non-adaptive \SSOT protocol in \figureref{fig:Consnssot} securely realizes the  functionality of \figureref{fig:nssotfunc} in the presence of semi-honest adversaries, provided that:
%	\begin{itemize}\addtolength{\itemsep}{-6pt}
%		\item  a secure 1-out-of-n \SSOT protocol in \figureref{fig:Cons1ssot} 
%	\item three random hash $\{H_1, H_2, H_3\}: \{ 0,1\}^* \to [1.2m]$ 
%	\item  max. stash size $s$, and max. number of items in $\SS$'s bin $\gamma$ as defined in Table~\ref{tbl:params}
%	\end{itemize}
%\end{theorem}
%
%\begin{proof}
%The proof of security of the non-adaptive \SSOT construction is very similar to that of  \theoremref{thm:sotpriv}. Most importantly,
%\end{proof}
 
\section{PSI}
\label{sec:psi}
The main application of \OPPRF is to improve the performance of semi-honest-secure \textbf{multi-party private set intersection (multi-party PSI)}. 
\subsection{Collusion-free Three-Party Private Set Intersection case}
\label{sect:3psi-construction}
We first consider a simple case of the multi-party PSI: the protocol for three parties. This protocol is presented in Figure~\ref{fig:3psi}. 

\begin{figure}[h]\centering
\framebox{
    \begin{minipage}{0.95\linewidth}
		\todo{.....} \\
		\todo{.....} use \todo{\batchOPRF} and \todo{\SSOT .......} \\
		\todo{.....} 
    \end{minipage}
}
\caption{The Three-Party PSI protocol}
\label{fig:3psi}
\end{figure} %subsection of main contruction
\subsection{Private Set Intersection of Three or greater}
\label{sect:npsi-construction}

Similar to the scheme of the previous section ~\ref{sect:3psi-construction}, the following psi protocol is secure for more than three parties if we allow \todo{ $P*$ and one party $P_i$ colluded}. 

\begin{figure}[h]\centering
\framebox{
    \begin{minipage}{0.95\linewidth}
		\todo{.....} \\
		\todo{.....} use \todo{\batchOPRF} and \todo{\SSOT .......} \\
		\todo{.....} 
    \end{minipage}
}
\caption{The PSI protcol for more than 3 parties in cases that we allow \todo{$P*$ and one party $P_i$ colluded}}
\label{fig:npsicollud}
\end{figure}

\subsubsection{Problem of previous scheme}
Problem is that $P^*$ and two neighbor parties of the party $P_i$ colludes, they can see incoming and outgoing values of $P_i$ \todo{........} To handle this problem, each party should have their own secret value. We describe a \todo{"`share distribution"' protocol} in Section ~\ref{sect:share}\todo{........}

\subsubsection{Share distribution}
\label{sect:share}

		\todo{.....} \\
		\todo{.....} call \todo{\SSOT} \\
		\todo{.....} 

\subsubsection{The share distribution protocol}
\begin{figure}[h]\centering
\framebox{
    \begin{minipage}{0.95\linewidth}
		\todo{.....} \\
		\todo{.....} use \todo{\batchOPRF}, \todo{\SSOT}, and \todo{ShareDistribtion .......} \\
		\todo{.....} 
    \end{minipage}
}
\caption{The \bf{collusion-free} PSI protcol for more than 3 parties}
\label{fig:npsicollud}
\end{figure}

\subsubsection{Optimization}
Our multi-party protocol is \todo{\textbf{faster} if we allow upto $n-1$ parties colluded}, where $n$ is the number of parties in total. This is because we do not need to call \SSOT in the share distribution protocol  ~\ref{sect:share} %subsection of main contruction
\iffullversion
\section{Implementation \& Performance}
\else
\section{Implementation \& \\ Performance}
\fi
\label{sec:performance}

We implemented our PSI protocol and report on its performance in comparison with \todo{...}
Our complete implementation is available on GitHub: \todo{\url{}}.




\section*{Acknowledgments}
%We thank Peter Rindal for contributing libraries and helpful suggestions to our protocol implementation. We also thank Michael Zohner for answering our many questions about the implementation of \cite{DBLP:conf/uss/Pinkas0SZ15}. Finally, we thank the anonymous CCS reviewers for their helpful feedback.

%The first author is supported by the Office of Naval Research (ONR)contract number N00014-14-C-0113.  The second author is supported byNSF Grants CNS-1350619 and CNS-1414119, in part by the Defense Advanced Research Projects Agency (DARPA) and the U.S. Army Research Office under contracts W911NF-15-C-0226, and an MIT Translational Fellowship. The third and fourth authors are supported by NSF award
%1149647 and a Google research award.  This work was initiated while
%the first three authors were visiting the Simons Institute for the
%Theory of Computing, supported by the Simons Foundation and by the
%DIMACS/Simons Collaboration in Cryptography through NSF grant
%\#CNS-1523467.


\iffullversion
    \bibliographystyle{alpha}
    \bibliography{bib,abbrev2,crypto_crossref}
\else

    %{
		%\small
    %\bibliographystyle{acm}
		\bibliographystyle{alpha}
    \bibliography{bib,abbrev2,crypto_crossref}
    %}
\fi
\end{document}
