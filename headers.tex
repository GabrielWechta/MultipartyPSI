\usepackage{xspace,amsmath,amsfonts,amssymb,hyperref,tikz,multirow}


\newcommand{\func}[1]{\ensuremath{\mathcal{F}_{\textsf{#1}}}\xspace}
\newcommand{\adv}{\ensuremath{\mathcal{A}}\xspace}

%%%%%%%%%%%%%%%%%%%%

\newcommand{\etal}{{\sl et~al.~}}
\newcommand{\eg}{{\sl e.g.}}
\newcommand{\ie}{{\sl i.e.}}
\newcommand{\apriori}{{\sl a~priori\/}\xspace}

\newcommand{\secpar}{\kappa}

\newcommand{\batchOPRF}{\textsf{BaRK-OPRF}\xspace}
\hyphenation{BaRK-OPRF}

\newcommand{\SSOT}{\colourblue{Private/Blind} \textsf{KS}\xspace}
\hyphenation{SSOT}

\newcommand{\OPPRF}{\textsf{OPPRF}\xspace}
\hyphenation{OPPRF}

\newcommand{\KS}{\textsf{Private KS}\xspace}
\hyphenation{KS}

\newcommand{\Enc}{\mathsf{Enc}}
\newcommand{\Dec}{\mathsf{Dec}}

%%%%%%%%%%%%%%%%%%%%%%

\newcommand{\rs}{\boldsymbol{r}}
\newcommand{\ts}{\boldsymbol{t}}
\newcommand{\us}{\boldsymbol{u}}
\newcommand{\qs}{\boldsymbol{q}}


%%%%%%%%%%%%%%%%%%%%%%

\newcommand{\algo}[1]{\ensuremath{\text{\sf #1}}\xspace}
\newcommand{\command}[1]{\ensuremath{\text{\sc #1}}\xspace}

%%%%%%%%%%%%%%%%%%%%

%% comment out for LNCS
\iffullversion
\usepackage{amsthm}
\newtheorem{theorem}{Theorem}
\newtheorem{definition}[theorem]{Definition}
\newtheorem{claim}[theorem]{Claim}
\newtheorem{lemma}[theorem]{Lemma}
\newtheorem{corol}[theorem]{Corollary}
\newtheorem{assumption}[theorem]{Assumption}
\newtheorem{obs}[theorem]{Observation}
\newtheorem{conj}[theorem]{Conjecture}
\fi

\newenvironment{proofof}[1]{\begin{proof}[Proof of #1.]}{\end{proof}}
\newenvironment{proofsketch}{\begin{proof}[Proof Sketch]}{\end{proof}}


%%%%%%%%%%%%%%%%%%%%%%%%%

\newcommand{\namedref}[2]{\hyperref[#2]{#1~\ref*{#2}}}
%% if you don't like it, use this instead:
%\newcommand{\namedref}[2]{#1~\ref{#2}}
\newcommand{\chapterref}[1]{\namedref{Chapter}{#1}}
\newcommand{\sectionref}[1]{\namedref{Section}{#1}}
\newcommand{\theoremref}[1]{\namedref{Theorem}{#1}}
\newcommand{\propositionref}[1]{\namedref{Proposition}{#1}}
\newcommand{\definitionref}[1]{\namedref{Definition}{#1}}
\newcommand{\corollaryref}[1]{\namedref{Corollary}{#1}}
\newcommand{\obsref}[1]{\namedref{Observation}{#1}}
\newcommand{\lemmaref}[1]{\namedref{Lemma}{#1}}
\newcommand{\claimref}[1]{\namedref{Claim}{#1}}
\newcommand{\figureref}[1]{\namedref{Figure}{#1}}
\newcommand{\subfigureref}[2]{\hyperref[#1]{Figure~\ref*{#1}#2}}
\newcommand{\equationref}[1]{\namedref{Equation}{#1}}
\newcommand{\appendixref}[1]{\namedref{Appendix}{#1}}


\definecolor{darkred}{rgb}{0.5, 0, 0} 
\definecolor{darkgreen}{rgb}{0, 0.5, 0} 
\definecolor{darkblue}{rgb}{0,0,0.5} 

\hypersetup{
    colorlinks=true,
    linkcolor=darkred,
    citecolor=darkgreen,
    urlcolor=darkblue   
}


%%%%%%%%%%%%%%%%%%%%%%%%%%%%%

\newcommand{\todo}[1]{%
    \mbox{}% prevent marginpar from being on previous paragraph
%    \marginpar{%
%        \colorbox{red!80!black}{\textcolor{white}{to-do}}%
%        \vspace*{-22pt}% hack!
%    }%
    \textcolor{red}{#1}%
}

\newcommand{\colourblue}[1]{%
	\mbox{}% prevent marginpar from being on previous paragraph
	%    \marginpar{%
	%        \colorbox{red!80!black}{\textcolor{white}{to-do}}%
	%        \vspace*{-22pt}% hack!
	%    }%
	\textcolor{blue}{#1}%
}

    \renewcommand{\topfraction}{0.9}    % max fraction of floats at top
    \renewcommand{\bottomfraction}{0.8} % max fraction of floats at bottom


    %   Parameters for TEXT pages (not float pages):
    \setcounter{topnumber}{2}
    \setcounter{bottomnumber}{2}
    \setcounter{totalnumber}{4}     % 2 may work better
    \setcounter{dbltopnumber}{2}    % for 2-column pages
    \renewcommand{\dbltopfraction}{0.9} % fit big float above 2-col. text
    \renewcommand{\textfraction}{0.07}  % allow minimal text w. figs
    %   Parameters for FLOAT pages (not text pages):
    \renewcommand{\floatpagefraction}{0.7}      % require fuller float pages
        % N.B.: floatpagefraction MUST be less than topfraction !!
    \renewcommand{\dblfloatpagefraction}{0.7}   % require fuller float pages



\newif\ifsubmission
\submissionfalse

\ifsubmission
  \newcommand{\ch}[1]{#1}
  \newcommand{\del}[1]{}
\else
  \newcommand{\del}[1]{\textcolor{magenta}{[to be deleted: {\em {#1}}]}}
  \newcommand{\ch}[1]{\textcolor{blue}{#1}} 
\fi
